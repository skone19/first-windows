\documentclass{beamer}
\usetheme{Copenhagen}
\author{Bert}
\title{A tale of two primes}

\begin{document}

\begin{frame}
\frametitle{Sets}
A \alert{set} is a collection of objects.\uncover<2->{ For example:
\[
Z=\{\text{cow},\text{pig},\text{elephant}\}.
\]}
\uncover<3->{We call the objects in $Z$ the \alert{elements} of $Z$.}\uncover<4->{ We write
\[
\text{cow} \in Z
\]}
\uncover<5->{with ``cow is an element of $Z$''.}\uncover<6->{ Frequently encountered sets are}
\[
\begin{split}
\uncover<7->{\mathbb{N}}\uncover<8->{ = \{1,2,3,\ldots \}}&\uncover<9->{ \qquad (\text{``natural numbers''})\\}\uncover<10->{\mathbb{Z}}\uncover<11->{ = \{\ldots,-2,-1,0,1,2,\ldots \}}&\uncover<12->{ \qquad (\text{``integer numbers''})\\}\uncover<13->{\mathbb{Q}}\uncover<14->{ = \{p/q : p,q\in\mathbb{Z} \text{ and } q\neq 0\}}&\uncover<15->{\qquad (\text{``rational numbers''})\\}\uncover<16->{\mathbb{R}}\uncover<17->{ = \{\text{decimal numbers}\}}&\uncover<18->{\quad\qquad (\text{``real numbers''})}
\end{split}
\]
\end{frame}

\end{document}